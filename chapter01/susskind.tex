\documentclass{amsart}
\usepackage{graphicx,float}
\usepackage{amsmath,xfrac,amssymb}
\usepackage{tikz}
\usetikzlibrary{arrows,shapes,positioning}
\usetikzlibrary{decorations.markings}

\tikzstyle arrowstyle=[scale=1]
\tikzstyle directed=[postaction={decorate,decoration={markings,
    mark=at position .65 with {\arrow[arrowstyle]{stealth}}}}]
\tikzstyle reverse directed=[postaction={decorate,decoration={markings,
    mark=at position .65 with {\arrowreversed[arrowstyle]{stealth};}}}]

\title{Quantum Field Theory Problem Sets}
\author{John Bortins}

\begin{document}

\maketitle{}



\section*{Exercise 1.6}
Given \[Z_0[J]=\exp\left(-\frac{1}{2}\int d^4 x\, d^4 y\, J(x) \Delta(x-y) J(y)\right)\]
where \[\Delta(x)=\Delta(-x)\]
Figure the functional derivative
\[\delta Z[J][h]=\frac{d}{d\epsilon}Z[J+\epsilon h]\]
Look at the integrand alone
\[(J(x)+\epsilon h) \Delta(x-y) (J(y)+\epsilon h)\]
\[(J(x)J(y)+\epsilon h J(x)+\epsilon h J(y)+\epsilon^2 h) \Delta(x-y)\]
Take the derivative with respect to $\epsilon$
\[J(x) \Delta(x-y)h + J(y) \Delta(x-y)h\]
\[\delta Z[J][h]=-\frac{1}{2}Z_0[J]\left(\int d^4 x\,  J(x) \Delta(x-y)h +\int d^4 y\,J(y) \Delta(x-y)h\right)\]
\[\delta Z[J][h]=-\frac{1}{2}Z_0[J]\left(\int d^4 x\,  J(x) \Delta(x-y)\delta(x-y) +\int d^4 y\,J(y) \Delta(x-y)\delta(x-y)\right)\]
Use \[\Delta(x)=\Delta(-x)\]
\[\delta Z[J][h]=-\frac{1}{2}Z_0[J]\left(\int d^4 x\,  J(x) \Delta(x-y)\delta(y-x) +\int d^4 y\,J(y) \Delta(x-y)\delta(x-y)\right)\] change variables
\[\delta Z[J][h]=-\left(\int d^4 y\,J(y) \Delta(y-x)\delta(x-y)\right)Z_0[J]\]
\[\frac{\delta Z[J]}{\delta J[z_1]}=-\left[\int d^4 y\, \Delta(z_1-y)J(y)\right]Z_0[J] \qquad \blacksquare\]

\section*{Exercise 10.1}

\end{document}
