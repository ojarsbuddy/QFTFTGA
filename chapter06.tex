\documentclass{article}
\usepackage{amsmath,amssymb,physics,xfrac}
\usepackage{standalone}
\usepackage[left=1.5cm,top=1.5cm,right=1.5cm,bottom=1.5cm]{geometry}

 
\title{Quantum Field Theory Problem Sets}
\author{John Bortins}
 
\begin{document}
 
\maketitle{}

\section*{Notes}

Relativistic quantum mechanics and the Klein-Gordon equation. Negaive energy states associated with positive energies of antiparticles.
 
\section*{Problem 6.1}

\begin{align*}
\qq*{Given}\mathcal{L}&=\frac{1}{2}(\partial_\mu \phi)^2 - \frac{1}{2} m^2 \phi^2\\
\qq*{From }\frac{\partial \mathcal{L}}{\partial \phi}-\partial_\mu  \left(\frac{\partial \mathcal{L}}{\partial (\partial_\mu \phi)}\right) &= 0 \qq{Euler-LaGrange in 4-vector form}\\
\qq*{We get}-m^2 \phi + \partial_\mu (\partial_\mu \phi)&= 0\\
\qq*{Hence}(\partial^2_\mu-m^2) \phi&=0\qquad\blacksquare
\end{align*}
 
The conjugate momentum is \[\pi=\frac{\partial \mathcal{L}}{\partial (\partial_0 \phi)}\] which works out to
\[\pi=\partial_0 \phi\]

The Hamiltonian is \[\mathcal{H}=\pi \partial_0 \phi - \mathcal{L}\] which works out to
\[\mathcal{H}=(\partial_0 \phi)^2-\frac{1}{2}(\partial_\mu \phi)^2 + \frac{1}{2} m^2 \phi^2\]

So we might say \[\mathcal{H}=\frac{1}{2}(\dot{\phi}^2-(\nabla  \phi)^2 +  m^2 \phi^2)\]
\end{document}