\documentclass{article}
\usepackage{amsmath,amssymb,physics,xfrac,nicefrac}
\usepackage{standalone}
\usepackage[left=1.5cm,top=1.5cm,right=1.5cm,bottom=1.5cm]{geometry}

 
\title{Quantum Field Theory Problem Sets}
\author{John Bortins}
 
\begin{document}
 
\maketitle{}

\section*{Notes}

Got some work to do on these problems.
 
\section*{Problem 9.1}

\begin{align*}
\qq*{Given the translation operator}\hat{U}(\vb{a})&=e^{-\text{i}\vb{\hat{p}}\cdot\vb{a}}\\
\qq*{Construct the derivative}\pdv{\hat{U}(\vb{a})}{\vb{a}}&=-\text{i}\vb{\hat{p}}e^{-\text{i}\vb{\hat{p}}\cdot\vb{a}}\\
\qq*{Shuffle the factors}\vb{\hat{p}}&=-\frac{1}{\text{i}}\pdv{\hat{U}(\vb{a})}{\vb{a}}e^{\text{i}\vb{\hat{p}}\cdot\vb{a}}\\
\qq*{Hence the generators are given by}\vb{\hat{p}}&=\eval{-\frac{1}{\text{i}}\pdv{\hat{U}(\vb{a})}{\vb{a}}}_{\vb{a}=0}\qquad\blacksquare
\end{align*}



\section*{Problem 9.2}

\[\qq*{The generators of the Lorentz transformations are given by}K^j=\eval{\frac{1}{\text{i}}\pdv{\hat{D}(\phi^j)}{\phi^j}}_{\phi^j=0}.\]
\[ \qq*{Take the} D(\phi^i) \qq{to be the Lorentz transformation matrices}\vb{\Lambda}(\phi^i).\]
\begin{align*}
\qq*{For the $x$-direction}D(\phi^1)&=\vb{\Lambda}(\phi^1)=\begin{pmatrix}
\cosh{\phi^1}&\sinh{\phi^1}&0&0\\
\sinh{\phi^1}&\cosh{\phi^1} & 0&0 \\ 
0 & 0&1&0\\
0&0&0&1
\end{pmatrix}\\
K^1=\eval{\frac{1}{\text{i}}\pdv{\hat{D}(\phi^1)}{\phi^1}}_{\phi^1=0}&=\eval{\frac{1}{\text{i}}\begin{pmatrix}
		\sinh{\phi^1}&\cosh{\phi^1}&0&0\\
		\cosh{\phi^1}&\sinh{\phi^1} & 0&0 \\ 
		0 & 0&0&0\\
		0&0&0&0
		\end{pmatrix}}_{\phi^1=0}=-\text{i}\begin{pmatrix}
		0&1&0&0\\
		1&0 & 0&0 \\ 
		0 & 0&0&0\\
		0&0&0&0
		\end{pmatrix}\qquad\blacksquare
\end{align*}\begin{align*}
\qq*{For the $y$-direction}D(\phi^2)&=\vb{\Lambda}(\phi^2)=\begin{pmatrix}
\cosh{\phi^2}&0&\sinh{\phi^2}&0\\
0&1 & 0&0 \\ 
\sinh{\phi^2} & 0&\cosh{\phi^2}&0\\
0&0&0&1
\end{pmatrix}\\
K^2=\eval{\frac{1}{\text{i}}\pdv{\hat{D}(\phi^2)}{\phi^2}}_{\phi^2=0}&=\eval{\frac{1}{\text{i}}\begin{pmatrix}
\sinh{\phi^2}&0&\cosh{\phi^2}&0\\
0&1 & 0&0 \\ 
\cosh{\phi^2} & 0&\sinh{\phi^2}&0\\
0&0&0&1
		\end{pmatrix}}_{\phi^2=0}=-\text{i}\begin{pmatrix}
		0&0&1&0\\
		0 & 0&0&0\\
		1&0 & 0&0 \\ 
		0&0&0&0
		\end{pmatrix}\qquad\blacksquare
\end{align*}
\begin{align*}
\qq*{For the $z$-direction}D(\phi^3)&=\vb{\Lambda}(\phi^3)=\begin{pmatrix}
\cosh{\phi^3}&0&0&\sinh{\phi^3}\\
0&1 & 0&0 \\ 
0 & 0&1&0\\
\sinh{\phi^3}&0&0&\cosh{\phi^3}
\end{pmatrix}\\
K^3=\eval{\frac{1}{\text{i}}\pdv{\hat{D}(\phi^3)}{\phi^3}}_{\phi^3=0}&=\eval{\frac{1}{\text{i}}\begin{pmatrix}
		\sinh{\phi^3}&0&0&\cosh{\phi^3}\\
0&1 & 0&0 \\ 
0 & 0&1&0\\
\cosh{\phi^3}&0&0&\sinh{\phi^3}
		\end{pmatrix}}_{\phi^3=0}=-\text{i}\begin{pmatrix}
		0&0&0&1\\
		0 & 0&0&0\\
		0&0&0&0 \\
		1&0 & 0&0 
		\end{pmatrix}\qquad\blacksquare
\end{align*}

\[\qq*{Recall}\dv{x}(\sinh x)=\cosh x\qand\dv{x}(\cosh x)=\sinh x.\]
\[\qq*{Note that}\sinh 0=0\qand\cosh 0=1.\]



\section*{Problem 9.3}

Show the Lorentz transformation for an infinitesmal boost $v^j$ along $x^j$-axis, \emph{i.e.}, $x^j+v^j t$. The Roman index denotes a component of a three-vector. Consider the Lorentz transformation matrix ${\vb{\Lambda}^\mu}_\nu\equiv\pdv*{\bar{x}^\mu}{x^\nu}$. The Lorentz factor has MacLaurin series $\gamma=1+\frac{1}{2}\beta^2+\frac{3}{8}\beta^4+\frac{5}{16}\beta^6+\cdots$. For an infinitesimal boost $v^j $ we have $\beta\ll 1 \implies\gamma\to 1$. 
\begin{align*}
	\bar{x}^j&=\gamma(x^j+v^j t)\\
	\pdv*{\bar{x}^j}{x^j}&=\gamma\to 1\\
	\pdv*{\bar{x}^j}{x^{i\neq j}}&=0\\
	\pdv*{\bar{x}^j}{x^0}&=	\gamma v^j\to v^j\\
	\bar{x}^0&=\gamma(x^0v^j x^j)\\
	\pdv*{\bar{x}^0}{x^0}&=\gamma\to 1\\
	\pdv*{\bar{x}^0}{x^j}&=\gamma v^j\to v^j\\
\end{align*}

\[ \qq*{Pop in the}\pdv*{\bar{x}^\mu}{x^\nu}\qq{to find}{\vb{\Lambda}^\mu}_\nu=\begin{pmatrix}1&v^1&v^2&v^3\\v^1&1&0&0\\v^2&0&1&0\\v^3&0&0&1\\
\end{pmatrix}\]

Show the Lorentz transformation for an infinitesimal rotation $\theta^j$ about $x^j$. 

\begin{align*}
	\qq*{Consider the rotation around}x\qquad
	\bar{y}&=y\cos\theta-z\sin\theta\\
	\bar{z}&=y\sin\theta+z\cos\theta
\end{align*}

Recall MacLaurin series $\sin\theta=\theta-\frac{\theta^3}{3!}+\frac{\theta^5}{5!}-\cdots$ and $\cos\theta=1-\frac{\theta^2}{2!}+\frac{\theta^4}{4!}-\cdots$.
For an infinitesimal rotation we have $\theta\ll1\implies \sin\theta\to\theta, \cos\theta\to 1$.

\begin{align*}
	\qq{For an infinitesimal rotation around}x^1\qquad \bar{x}^1 &\to x^1\\
	\bar{x}^2 &\to x^2 -x^3 \theta^1\\
	\bar{x}^3 &\to x^2 \theta^1 +x^3\\
	\qq{For an infinitesimal rotation around}x^2\qquad \bar{x}^2 &\to x^2\\
	\bar{x}^3 &\to x^3 -x^1 \theta^2\\
	\bar{x}^1 &\to x^3 \theta^2 +x^1\\
	\qq{For an infinitesimal rotation around}x^3\qquad \bar{x}^3 &\to x^3\\
	\bar{x}^1 &\to x^1 -x^2 \theta^3\\
	\bar{x}^2 &\to x^1 \theta^3 +x^2\\
	\qq{And} \bar{x}^0 &\to x^0\\
\end{align*}

\[ \qq*{Pop in the}\pdv*{\bar{x}^\mu}{x^\nu}\qq{to find}{\vb{\Lambda}^\mu}_\nu=\begin{pmatrix}1&0&0&0\\0&1&-\theta^3&\theta^2\\0&\theta^3&1&-\theta^1\\0&-\theta^2&\theta^1&1\\
\end{pmatrix}\]
This matches the sense of Eq. 9.2 rather than the answer in the book.

Show that a generalized infinitesimal Lorentz transformation can be written as $x'^\mu={\Lambda^\mu}_\nu x^\nu$ where $\vb{\Lambda}=\vb{1}+\vb{\omega}$.
\[ {\vb{\omega}^\mu}_\nu=\begin{pmatrix}1&v^1&v^2&v^3\\v^1&1&-\theta^3&\theta^2\\v^2&\theta^3&1&-\theta^1\\v^3&-\theta^2&\theta^1&1\\
\end{pmatrix}\]







\[\omega^{\mu\nu}={\omega^\mu}_\lambda g^{\lambda\nu}=\begin{pmatrix}0&v^1&v^2&v^3\\v^1&0&-\theta^3&\theta^2\\v^2&\theta^3&0&-\theta^1\\v^3&-\theta^2&\theta^1&0\end{pmatrix}
\begin{pmatrix}1&0&0&0\\0&-1&0&0\\0&0&-1&0\\0&0&0&-1\end{pmatrix}=
\begin{pmatrix}0&-v^1&-v^2&-v^3\\v^1&0&\theta^3&-\theta^2\\v^2&-\theta^3&0&\theta^1\\v^3&\theta^2&-\theta^1&0\end{pmatrix}\]

\[\omega_{\mu\nu}=g_{\mu\lambda}{\omega^\lambda}_\nu =\begin{pmatrix}1&0&0&0\\0&-1&0&0\\0&0&-1&0\\0&0&0&-1\end{pmatrix}
\begin{pmatrix}0&v^1&v^2&v^3\\v^1&0&-\theta^3&\theta^2\\v^2&\theta^3&0&-\theta^1\\v^3&-\theta^2&\theta^1&0\end{pmatrix}=
\begin{pmatrix}0&v^1&v^2&v^3\\-v^1&0&\theta^3&-\theta^2\\-v^2&-\theta^3&0&\theta^1\\-v^3&\theta^2&-\theta^1&0\end{pmatrix}\]

\[\begin{pmatrix}0&-v^1&-v^2&-v^3\\v^1&0&\theta^3&-\theta^2\\v^2&-\theta^3&0&\theta^1\\v^3&\theta^2&-\theta^1&0\end{pmatrix}^T=\begin{pmatrix}0&v^1&v^2&v^3\\-v^1&0&-\theta^3&\theta^2\\-v^2&\theta^3&0&-\theta^1\\-v^3&-\theta^2&\theta^1&0\end{pmatrix}=-\begin{pmatrix}0&-v^1&-v^2&-v^3\\v^1&0&\theta^3&-\theta^2\\v^2&-\theta^3&0&\theta^1\\v^3&\theta^2&-\theta^1&0\end{pmatrix}\]

\[\begin{pmatrix}0&v^1&v^2&v^3\\-v^1&0&\theta^3&-\theta^2\\-v^2&-\theta^3&0&\theta^1\\-v^3&\theta^2&-\theta^1&0\end{pmatrix}^T=\begin{pmatrix}0&-v^1&-v^2&-v^3\\v^1&0&-\theta^3&\theta^2\\v^2&\theta^3&0&-\theta^1\\v^3&-\theta^2&\theta^1&0\end{pmatrix}=-\begin{pmatrix}0&v^1&v^2&v^3\\-v^1&0&\theta^3&-\theta^2\\-v^2&-\theta^3&0&\theta^1\\-v^3&\theta^2&-\theta^1&0\end{pmatrix}\]

Clearly the transposes of the resultant matrices are the negatives of the resultant matrices. Hence the resultant matrices are skew-symmetric, which is otherwise known as antisymmetric.


 
\section*{Problem 9.4}

\[\qq*{Given } f(x^\prime) = f(x+a+\omega x) = f(x)+a^\mu \partial_\mu f(x)+{\omega^\mu}_\nu x^\nu \partial_\mu f(x)\]

\begin{align*}
	\qq*{We know these things}\omega^{\mu\nu}&={\omega^\mu}_\lambda g^{\lambda\nu}\\
	\omega_{\mu\nu}&= g_{\mu\lambda} {\omega^\lambda}_\nu\\
	g^{\mu\lambda}g_{\lambda\nu}&= I\\
	A_\mu&=g_{\mu\nu}A^\nu\\
	A^\mu&=g^{\mu\nu}A_\nu\\
	{\omega^\mu}_\nu&=\frac{1}{2}[{\omega^\mu}_\nu+{{\omega^\mu}_\nu}^T]+\frac{1}{2}[{\omega^\mu}_\nu-{{\omega^\mu}_\nu}^T]\\
	&=\frac{1}{2}[{\omega^\mu}_\nu+{\omega_\mu}^\nu]+\frac{1}{2}[{\omega^\mu}_\nu-{\omega_\mu}^\nu]
\end{align*}




Examine \[{\omega^\mu}_\nu x^\nu \partial_\mu=\sum^3_{\mu=0}\sum^3_{\nu=0}{\omega^\mu}_\nu x^\nu \partial_\mu\]
\begin{multline*}
{\omega^\mu}_\nu x^\nu \partial_\mu=
v^1 x^1 \partial_0+v^2 x^2 \partial_0+v^3 x^3 \partial_0+
v^1 x^0 \partial_1+\theta^3 x^2 \partial_1-\theta^2 x^3 \partial_1\\+ 
v^2 x^0 \partial_2-\theta^3 x^1 \partial_2+\theta^1 x^3 \partial_2+
v^3 x^0 \partial_3+\theta^2 x^1 \partial_3-\theta^1 x^2 \partial_3
\end{multline*}

Examine \[\omega_{\mu\nu} (x^\mu \partial^\nu-x^\nu \partial^\mu)=\sum^3_{\mu=0}\sum^3_{\nu=0}\omega_{\mu\nu} x^\mu \partial^\nu-\sum^3_{\mu=0}\sum^3_{\nu=0}\omega_{\mu\nu} x^\nu \partial^\mu\]
\begin{multline*}
\omega_{\mu\nu} (x^\mu \partial^\nu-x^\nu \partial^\mu)=
v^1 x^0 \partial^1+v^2 x^0 \partial^2+v^3 x^0 \partial^3
-v^1 x^1 \partial^0-\theta^3 x^1 \partial^2+\theta^2 x^1 \partial^3\\ 
-v^2 x^2 \partial^0+\theta^3 x^2 \partial^1-\theta^1 x^2 \partial^3
-v^3 x^3 \partial^0-\theta^2 x^3 \partial^1+\theta^1 x^3 \partial^2\\-
(v^1 x^1 \partial^0+v^2 x^2 \partial^0+v^3 x^3 \partial^0
-v^1 x^0 \partial^1-\theta^3 x^2 \partial^1+\theta^2 x^3 \partial^1\\ 
-v^2 x^0 \partial^2+\theta^3 x^1 \partial^2-\theta^1 x^3 \partial^2
-v^3 x^0 \partial^3-\theta^2 x^1 \partial^3+\theta^1 x^2 \partial^3)\end{multline*}
\begin{multline*}
\omega_{\mu\nu} (x^\mu \partial^\nu-x^\nu \partial^\mu)=
v^1 x^0 \partial^1+v^2 x^0 \partial^2+v^3 x^0 \partial^3
-v^1 x^1 \partial^0-\theta^3 x^1 \partial^2+\theta^2 x^1 \partial^3\\ 
-v^2 x^2 \partial^0+\theta^3 x^2 \partial^1-\theta^1 x^2 \partial^3
-v^3 x^3 \partial^0-\theta^2 x^3 \partial^1+\theta^1 x^3 \partial^2\\
-v^1 x^1 \partial^0-v^2 x^2 \partial^0-v^3 x^3 \partial^0
+v^1 x^0 \partial^1+\theta^3 x^2 \partial^1-\theta^2 x^3 \partial^1\\ 
+v^2 x^0 \partial^2-\theta^3 x^1 \partial^2+\theta^1 x^3 \partial^2
+v^3 x^0 \partial^3+\theta^2 x^1 \partial^3-\theta^1 x^2 \partial^3\end{multline*}

But $\partial^i=-\partial_i$ and $\partial^0=\partial_0$, therefore
\begin{multline*}
\omega_{\mu\nu} (x^\mu \partial^\nu-x^\nu \partial^\mu)=
-v^1 x^0 \partial_1-v^2 x^0 \partial_2-v^3 x^0 \partial_3
-v^1 x^1 \partial_0+\theta^3 x^1 \partial_2-\theta^2 x^1 \partial_3\\ 
-v^2 x^2 \partial_0-\theta^3 x^2 \partial_1+\theta^1 x^2 \partial_3
-v^3 x^3 \partial_0+\theta^2 x^3 \partial_1-\theta^1 x^3 \partial_2\\
-v^1 x^1 \partial_0-v^2 x^2 \partial_0-v^3 x^3 \partial_0
-v^1 x^0 \partial_1-\theta^3 x^2 \partial_1+\theta^2 x^3 \partial_1\\ 
-v^2 x^0 \partial_2+\theta^3 x^1 \partial_2-\theta^1 x^3 \partial_2
-v^3 x^0 \partial_3-\theta^2 x^1 \partial_3+\theta^1 x^2 \partial_3\end{multline*}

Finally
\[{\omega^\mu}_\nu x^\nu \partial_\mu=-\frac{1}{2}\omega_{\mu\nu} (x^\mu \partial^\nu-x^\nu \partial^\mu)\]

\end{document}