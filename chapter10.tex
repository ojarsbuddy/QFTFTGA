\documentclass{article}
\usepackage{amsmath,amssymb,physics,xfrac,nicefrac}
\usepackage{standalone}
\usepackage[left=1.5cm,top=1.5cm,right=1.5cm,bottom=1.5cm]{geometry}
\usepackage[indentfirst=true]{quoting}
\usepackage{xcolor, soul}
\definecolor{HighlightGray}{cmyk}{0,0,0,0.07}
\sethlcolor{HighlightGray}
 
\title{Quantum Field Theory Problem Sets}
\author{John Bortins}
 
\begin{document}

\maketitle{}

\section*{Problem 10.1}
\begin{quoting}
  \hl{Show that $[\phi(x),P^\alpha]=i\partial^\alpha \phi(x)$, where $P^\alpha$ is the conserved charges from spacetime translation (10.28).}
\end{quoting}

\begin{align*}
  \qq*{The energy-momentum tensor}T^{\mu\nu} & =\Pi^\mu\partial^\nu \phi-g^{\mu\nu}\mathcal{L}                       \\
  \qq*{The conserved charges}P^\alpha        & =\int\dd[3]{x}T^{0\alpha}                                             \\
                                             & =\int\dd[3]{x}\pqty{\Pi^0\partial^\alpha \phi-g^{0\alpha}\mathcal{L}} \\
  \qq*{Because}g^{00}=1\qquad P^0            & =\int\dd[3]{x}\pqty{\Pi^0\partial^0 \phi-\mathcal{L}}                 \\
  \qq*{Because}g^{0k}=0\qquad P^k            & =\int\dd[3]{x}\pqty{\Pi^0\partial^k \phi}                             \\
\end{align*}

\begin{align*}
  \qq*{Integration by parts} \dv{x}(uv) & =\dv{u}{x}v+u\dv{u}{x}                        \\
  \int \dv{x}(uv) \dd{x}                & = \int \dv{u}{x}v\dd{x}+\int u\dv{v}{x}\dd{x} \\    
  uv                                    & =\int v\dd{u}+\int u\dd{v}                    \\           
\end{align*}


\begin{align*}
  [\phi(x),P^\alpha] & =\phi(x)P^\alpha-P^\alpha \phi(x)                                                                                                                                                               \\
                     & =\phi(x)\int\dd[3]{x}T^{0\alpha}-\pqty{\int\dd[3]{x}T^{0\alpha}} \phi(x)                                                                                                                        \\
                     & =\phi(x)\int\dd[3]{x}\pqty{\Pi^0\partial^\alpha \phi-g^{0\alpha}\mathcal{L}}-\bqty{\int\dd[3]{x}\pqty{\Pi^0\partial^\alpha \phi-g^{0\alpha}\mathcal{L}}} \phi(x)\qq{next integrate by parts}    \\
                     & =\partial^\alpha \phi(x)\int\dd[3]{x}\pqty{\Pi^0\partial^\alpha \phi-g^{0\alpha}\mathcal{L}}+\phi(x)\pqty{\Pi^0\partial^\alpha \phi-g^{0\alpha}\mathcal{L}}                                     \\
                     & \qquad-\pqty{\Pi^0\partial^\alpha \phi-g^{0\alpha}\mathcal{L}}\phi(x)-\bqty{\int\dd[3]{x}\pqty{\Pi^0\partial^\alpha \phi-g^{0\alpha}\mathcal{L}}} \partial^\alpha\phi(x)                        \\
                     & =\partial^\alpha \phi(x)\int\dd[3]{x}\pqty{\Pi^0\partial^\alpha \phi-g^{0\alpha}\mathcal{L}}-\bqty{\int\dd[3]{x}\pqty{\Pi^0\partial^\alpha \phi-g^{0\alpha}\mathcal{L}}} \partial^\alpha\phi(x) \\
                     & \qquad +\phi(x)\Pi^0\partial^\alpha \phi-\phi(x)g^{0\alpha}\mathcal{L}-\pqty{\Pi^0\partial^\alpha \phi}\phi(x)+g^{0\alpha}\mathcal{L}\phi(x)                                                    \\
                     & =\phi(x)\Pi^0\partial^\alpha \phi-\pqty{\Pi^0\partial^\alpha \phi}\phi(x)-\phi(x)g^{0\alpha}\mathcal{L}+g^{0\alpha}\mathcal{L}\phi(x)                                                           \\
                     & =\phi(x)\Pi^0\partial^\alpha \phi-\pqty{\Pi^0\partial^\alpha \phi}\phi(x)=\bqty{\phi(x)\Pi^0-\Pi^0 \phi(x)}\partial^\alpha \phi=\bqty{\phi(x),\Pi^0}\partial^\alpha \phi                        \\
  [\phi(x),P^\alpha] & =i\partial^\alpha \phi   (x)                                                                                                                                                                    \\
                     & \qq{where}\Pi^0(x)=\pi(x)\qand\bqty{\phi(x),\pi(x)}=i\qq{recognizing the conjugate momentum}\blacksquare
\end{align*}



\section*{Problem 10.2}
\begin{quoting}
  \hl{Consider a system characterized by $N$ fields $\phi_1,\dots,\phi_N$. The Lagrangian density is then\\ $\mathcal{L}(\phi_1,\dots,\phi_N;\partial_\mu \phi_1,\dots,\partial_\mu \phi_N;x^\mu)$. Show that the Noether current is $J^\mu=\sum_\alpha \Pi_\alpha^\mu  D\phi^\alpha-W^\mu(x)$, (10.41),\\ where $D\mathcal{L}=\partial_\mu W^\mu$.}
\end{quoting}

\[\qq*{The change in the Lagranian density is}\delta\mathcal{L}=\pdv{\mathcal{L}}{\phi^\alpha}\delta\phi^\alpha+\pdv{\mathcal{L}}{(\partial_\mu \phi^\beta)}\delta(\partial_\mu \phi^\beta).\]
\[\qq*{Define}\Pi^\mu_\beta=\pdv{\mathcal{L}}{(\partial_\mu \phi^\beta)}\qcomma \qthen \delta\mathcal{L}=\pdv{\mathcal{L}}{\phi^\alpha}\delta\phi^\alpha+\Pi^\mu_\beta \delta(\partial_\mu \phi^\beta).\]
\[\qq*{Use}\delta(\partial_\mu \phi^\beta)=\partial_\mu (\delta\phi^\beta)\qand\partial_\mu (\Pi^\mu_\beta \delta\phi^\beta)=(\partial_\mu \Pi^\mu_\beta) \delta\phi^\beta+\Pi^\mu_\beta \partial_\mu (\delta\phi^\beta).\]
\[\qq*{Hence}\partial_\mu (\Pi^\mu_\beta \delta\phi^\beta)=(\partial_\mu \Pi^\mu_\beta) \delta\phi^\beta+\Pi^\mu_\beta \delta(\partial_\mu \phi^\beta)\implies\Pi^\mu_\beta \delta(\partial_\mu \phi^\beta)=\partial_\mu (\Pi^\mu_\beta \delta\phi^\beta)-(\partial_\mu \Pi^\mu_\beta) \delta\phi^\beta.\]
\[\qq*{Then}\delta\mathcal{L}=\pqty{\pdv{\mathcal{L}}{\phi^\alpha}-\partial_\mu \Pi^\mu_\alpha}\delta\phi^\alpha+\partial_\mu (\Pi^\mu_\beta \delta\phi^\beta).\]
\[\delta S=\int\dd[4]{x}\pqty{\pdv{\mathcal{L}}{\phi^\alpha}-\partial_\mu \Pi^\mu_\alpha}\delta\phi^\alpha=0\qand\pdv{\mathcal{L}}{\phi^\alpha}=\partial_\mu \Pi^\mu_\alpha.\]
\[\qq*{Say}\phi\qq{obeys the equations of motion}\qthen \mathcal{L}=\partial_\mu (\Pi^\mu_\beta \delta\phi^\beta).\]
\[\qq*{Recall}\delta\phi=D\phi\delta\lambda\qthen\mathcal{L}=\partial_\mu (\Pi^\mu_\beta D\phi^\beta) \delta\lambda.\]
\[\qq*{Introduce}W^\mu(x)\qq{such that}\delta\mathcal{L}=(\partial_\mu W^\mu)\delta\lambda\]
\[\qq*{Hence}\partial_\mu (\Pi^\mu_\beta D\phi^\beta-W^\mu)=0.\]
\[\qq*{Noether current} J^\mu=\Pi^\mu_\beta D\phi^\beta-W^\mu\qq{where}D\mathcal{L}=\partial_\mu W^\mu.\qquad\blacksquare\]

\section*{Problem 10.3}
\begin{quoting}
  \hl{For the Lagrangian $\mathcal{L}=\frac{1}{2}(\partial_\mu \phi)^2-\frac{1}{2}m^2\phi^2$, (10.42), evaluate $T^{\mu\nu}$ and show that $T^{00}$ agrees with what you would expect from the Hamiltonian for this Lagrangian. Show that $\partial_\mu T^{\mu\nu}=0$. Derive expressions for $P^0=\int\dd[3]{x}T^{00}$ and $P^i=\int\dd[3]{x}T^{0i}$.}
\end{quoting}
See \S 7.2, a massive scalar field.
\begin{align*}
  \qq*{A few definitions}\partial_\mu\equiv\pdv{x^\mu}=\pqty{\pdv{x^0},\pdv{x^1},\pdv{x^2},\pdv{x^3}}=\pqty{\pdv{t},\grad}=\pqty{\pdv{t},\pdv{x},\pdv{y},\pdv{z}}
\end{align*}

\begin{align*}
  \mathcal{L}(\phi,\partial_\mu \phi) & =\frac{1}{2}(\partial_\mu \phi)^2-\frac{1}{2}m^2\phi^2                                                                        \\
  \delta\mathcal{L}                   & =(\partial^\mu \phi)\delta(\partial_\mu \phi)-(m^2 \phi)\delta\phi                                                            \\
                                      & =(\partial^\mu \phi)(\partial_\mu \delta\phi)-(m^2 \phi)\delta\phi\qq{where}\delta(\partial_\mu \phi)=\partial_\mu \delta\phi \\
                                      & =\Pi^\mu(\partial_\mu \delta\phi)-(m^2 \phi)\delta\phi\qq{where}\Pi^\mu=\partial_\mu \phi                                     \\
                                      & =\partial_\mu (\Pi^\mu \delta\phi)-(\partial_\mu \Pi^\mu) \delta\phi-(m^2 \phi)\delta\phi                                     \\
                                      & =\partial_\mu (\Pi^\mu \delta\phi)-(\partial_\mu \Pi^\mu+m^2 \phi)\delta\phi                                                  \\
                                      & =    \partial_\mu (\Pi^\mu \delta\phi)-({\partial_\mu}^2 +m^2 )\phi\delta\phi                                                 \\
\end{align*}
\begin{align*}
  \qq*{The d'Alembertian operator}\partial^2=\partial^\mu \partial_\mu=\pdv[2]{t}-\laplacian=\pdv[2]{t}-\pdv[2]{x}-\pdv[2]{y}-\pdv[2]{z}
\end{align*}
\begin{align*}
  T^{\mu\nu} & = \Pi^\mu (\partial^\nu \phi)-g^{\mu\nu}\mathcal{L}                                                                                                \\
             & = (\partial_\mu \phi) (\partial^\nu \phi)-g^{\mu\nu}\frac{1}{2}(\partial \phi)^2+g^{\mu\nu}\frac{1}{2}m^2\phi^2                                    \\
  T^{0i}     & = (\partial_0 \phi) (\partial^i \phi)=\dot{\phi}\partial^i \phi                                                                                    \\
  T^{00}     & =\frac{1}{2}(\partial_0 \phi)^2+\frac{1}{2}m^2\phi^2\qq{where}g^{00}=1                                                                             \\
             & =\dot{ \phi}^2-\frac{1}{2}\dot{ \phi}^2-\frac{1}{2}\laplacian\phi+\frac{1}{2}m^2\phi^2  \qquad\blacksquare                                         \\
             & =\frac{1}{2}\dot{ \phi}^2-\frac{1}{2}\laplacian\phi+\frac{1}{2}m^2\phi^2\qq{which looks like the expected Hamiltonian density}  \qquad\blacksquare \\
\end{align*}

\begin{align*}
  \pdv{\mathcal{L}(\phi,\partial_\nu \phi)}{x_\nu}                                                               & =0                                                                                                                                                  \\
  \dv{\mathcal{L}}{x_\nu}=\partial^\nu \mathcal{L}                                                               & =\pdv{\mathcal{L}}{(\partial_\mu \phi)}\pdv{(\partial_\mu \phi)}{x_\nu}+\pdv{\mathcal{L}}{\phi}\pdv{\phi}{x_\nu}                                    \\
                                                                                                                 & =\pdv{\mathcal{L}}{(\partial_\mu \phi)}\partial^\nu (\partial_\mu \phi)+\pdv{\mathcal{L}}{\phi}\partial^\nu \phi                                    \\
                                                                                                                 & =\pdv{\mathcal{L}}{(\partial_\mu \phi)}\partial^\nu (\partial_\mu \phi)+\partial_\mu \pqty{\pdv{\mathcal{L}}{(\partial_\mu \phi)}}\partial^\nu \phi \\
                                                                                                                 & =\pdv{\mathcal{L}}{(\partial_\mu \phi)}\partial^\mu (\partial_\nu \phi)+\partial_\mu \pqty{\pdv{\mathcal{L}}{(\partial_\mu \phi)}}\partial^\nu \phi \\
                                                                                                                 & =\partial_\mu \pqty{\pdv{\mathcal{L}}{(\partial_\mu \phi)}\partial^\nu \phi}                                                                        \\
                                                                                                                 & =\partial_\mu g^{\mu\nu}\mathcal{L}                                                                                                                 \\
  \partial_\mu \pqty{\pdv{\mathcal{L}}{(\partial_\mu \phi)}\partial^\nu \phi}-\partial_\mu g^{\mu\nu}\mathcal{L} & =0                                                                                                                                                  \\
  \partial_\mu \pqty{\pdv{\mathcal{L}}{(\partial_\mu \phi)}\partial^\nu \phi- g^{\mu\nu}\mathcal{L}}             & =0                                                                                                                                                  \\
  \partial_\mu T^{\mu\nu}                                                                                        & =0           \qquad\blacksquare                                                                                                                     \\
\end{align*}

\begin{align*}
  P^0=\int\dd[3]{x}T^{00} & = \int\dd[3]{x}\pqty{\frac{1}{2}\dot{ \phi}^2-\frac{1}{2}\laplacian\phi+\frac{1}{2}m^2\phi^2}\qquad\blacksquare \\
  P^i=\int\dd[3]{x}T^{0i} & = \int\dd[3]{x}\dot{\phi}\partial^i \phi\qquad\blacksquare                                                      \\
\end{align*}



\section*{Problem 10.4}
\begin{quoting}
  \hl{For the Lagrangian $\mathcal{L}=-\frac{1}{4}F_{\mu\nu}F^{\mu\nu}=\frac{1}{2}(\vb{E}^2-\vb{B}^2)$, (10.43), show that $\Pi^{\sigma\rho}\equiv\pdv{\mathcal{L}}{(\partial_\sigma A_\rho)}=-F^{\sigma\rho}$, (10.44). Hence show that the energy-momentum tensor $T^\mu_\nu=\Pi^{\mu\sigma} \partial_\nu A_\sigma-\delta^\mu_\nu \mathcal{L}$, (10.45), can be written as\\ $T^{\mu\nu}=-F^{\mu\sigma}\partial^\nu A_\sigma+\frac{1}{4}g^{\mu\nu}F^{\alpha\beta}F_{\alpha\beta}$, (10.46). This tensor is not symmetric but we can symmetrize it by adding $\partial_\lambda X^{\lambda\mu\nu}$ where $X^{\lambda\mu\nu}=F^{\mu\lambda}A^\nu$. Show that $X^{\lambda\mu\nu}=-X^{\mu\lambda\nu}$. Show further that the symmetrized energy-momentum tensor $\tilde{T}^{\mu\nu}=T^{\mu\nu}+\partial_\lambda X^{\lambda\mu\nu}$ can be written $\tilde{T}^{\mu\nu}=F^{\mu\sigma}{F_\sigma}^\nu+\frac{1}{4}g^{\mu\nu}F^{\alpha\beta}F_{\alpha\beta}$, (10.47). Hence show that $\tilde{T}^{00}=\frac{1}{2}(\vb{E}^2+\vb{B}^2)$, the energy density in the electromagnetic field, and $\tilde{T}^{i0}=(\vb{E}\cross\vb{B})^i$, which is the Poynting vector and describes the energy flow.}
\end{quoting}

\begin{align*}
  \partial_\mu & \equiv\pdv{x^\mu}=\pqty{\pdv{x^0},\pdv{x^1},\pdv{x^2},\pdv{x^3}}=\pqty{\pdv{t},\grad}=\pqty{\pdv{t},\pdv{x},\pdv{y},\pdv{z}} \\
  \partial^\mu & =g^{\mu\nu}\partial_\nu=\pqty{\pdv{t},-\grad}=\pqty{\pdv{t},-\pdv{x},-\pdv{y},-\pdv{z}}                                      \\
  A^\mu        & = \pqty{V,\vb{A}}= \pqty{A^0,A^1,A^2,A^3} = \pqty{V,A_x,A_y,A_z}                                                             \\
  A_\mu        & = \pqty{V,-\vb{A}}= \pqty{A^0,-A^1,-A^2,-A^3}= \pqty{V,-A_x,-A_y,-A_z}                                                       \\
\end{align*}




\begin{align*}
  F_{\mu\nu} & =\partial_\mu A_\nu-\partial_\nu A_\mu=-\pqty{\partial_\nu A_\mu-\partial_\mu A_\nu} =-F_{\nu\mu} \\
  F^{\mu\nu} & =\partial^\mu A^\nu-\partial^\nu A^\mu=-\pqty{\partial^\nu A^\mu-\partial^\mu A^\nu} =-F^{\nu\mu} \\
\end{align*}

\begin{align*}
  \mathcal{L} & =-\frac{1}{4}F_{\mu\nu}F^{\mu\nu}                                                                                                                                              \\
              & =-\frac{1}{4}\pqty{\partial_\mu A_\nu - \partial_\nu A_\mu}\pqty{\partial^\mu A^\nu-\partial^\nu A^\mu}                                                                        \\
              & =-\frac{1}{4}\pqty{\partial_\mu A_\nu \partial^\mu A^\nu -\partial_\mu A_\nu \partial^\nu A^\mu - \partial_\nu A_\mu \partial^\mu A^\nu+\partial_\nu A_\mu \partial^\nu A^\mu} \\
              & =-\frac{1}{4}\pqty{\partial_\mu A_\nu \partial^\mu A^\nu -\partial_\mu A_\nu \partial^\nu A^\mu - \partial_\nu A_\mu \partial^\mu A^\nu+\partial_\nu A_\mu \partial^\nu A^\mu} \\
              & =-\frac{1}{2}\pqty{\partial_\mu A_\nu \partial^\mu A^\nu -\partial_\mu A_\nu \partial^\nu A^\mu} \qq{via relabeling}                                                           \\
\end{align*}

\begin{align*}
  \mathcal{L}                                                & =-\frac{1}{2}\pqty{\partial_\mu A_\nu \partial^\mu A^\nu -\partial_\mu A_\nu \partial^\nu A^\mu} \\
  \Pi^{\mu\nu}= \pdv{\mathcal{L}}{\pqty{\partial_\mu A_\nu}} & =-\pqty{\partial^\mu A^\nu -\partial^\nu A^\mu}=-F^{\mu\nu}  \qquad\blacksquare                  \\
\end{align*}

\begin{align*}
  T^\mu_\nu  & =\Pi^{\mu\sigma}\partial_\nu A_\sigma-\delta^\mu_\nu\mathcal{L}                                              \\
             & =-F^{\mu\sigma}\partial_\nu A_\sigma-\delta^\mu_\nu\mathcal{L}                                               \\
  T^{\mu\nu} & =T^\mu_\rho g^{\rho\nu}=\pqty{-F^{\mu\sigma}\partial_\rho A_\sigma-\delta^\mu_\rho\mathcal{L}}g^{\rho\nu}    \\
             & =-F^{\mu\sigma}\partial_\rho A_\sigma g^{\rho\nu}-\delta^\mu_\rho\mathcal{L}g^{\rho\nu}                      \\
             & =-F^{\mu\sigma}\partial^\nu A_\sigma -g^{\mu\nu} \mathcal{L}                                                 \\
             & =-F^{\mu\sigma}\partial^\nu A_\sigma +\frac{1}{4}g^{\mu\nu} F_{\alpha\beta}F^{\alpha\beta}                   \\
             & =-F^{\mu\sigma}\partial^\nu A_\sigma +\frac{1}{4}g^{\mu\nu} F^{\alpha\beta}F_{\alpha\beta}\qquad\blacksquare \\
\end{align*}

\begin{align*}
  X^{\lambda\mu\nu} & =F^{\mu\lambda}A^\nu =-F^{\lambda\mu}A^\nu=-X^{\mu\lambda\nu}\qquad\blacksquare \\
                    & =\pqty{\partial^\mu A^\lambda-\partial^\lambda A^\mu}A^\nu                      \\
                    & =-\pqty{-\partial^\mu A^\lambda+\partial^\lambda A^\mu}A^\nu                    \\
                    & =-\pqty{\partial^\lambda A^\mu-\partial^\mu A^\lambda}A^\nu                     \\
                    & =-F^{\lambda\mu}A^\nu                                                           \\
                    & =-X^{\mu\lambda\nu}\qquad\blacksquare
\end{align*}

\begin{align*}
  \tilde{T}^{\mu\nu} & =T^{\mu\nu} +\partial_\lambda X^{\lambda\mu \nu}                                                                                                                                  \\
                     & =-F^{\mu\sigma}\partial^\nu A_\sigma +\frac{1}{4}g^{\mu\nu} F^{\alpha\beta}F_{\alpha\beta} +\partial_\lambda \pqty{F^{\mu\lambda}A^\nu}                                           \\
                     & =-F^{\mu\sigma}\partial^\nu A_\sigma +\frac{1}{4}g^{\mu\nu} F^{\alpha\beta}F_{\alpha\beta} +\pqty{\partial_\sigma F^{\mu\sigma}}A^\nu  +F^{\mu\sigma}\pqty{\partial_\sigma A^\nu} \\
                     & =F^{\mu\sigma}\pqty{-\partial^\nu A_\sigma + {\partial_\sigma A^\nu}}+\frac{1}{4}g^{\mu\nu} F^{\alpha\beta}F_{\alpha\beta} +\pqty{\partial_\sigma F^{\mu\sigma}}A^\nu             \\
                     & =F^{\mu\sigma}{F_\sigma}^\nu+\frac{1}{4}g^{\mu\nu} F^{\alpha\beta}F_{\alpha\beta} \qq{where}0=\pqty{\partial_\sigma F^{\mu\sigma}}A^\nu  \qquad\blacksquare                       \\
\end{align*}

\begin{align*}
  F^{\mu\nu}                                             & =\Pmqty{0                                                                                                            & -E^1 & -E^2 & -E^3 \\E^1 &0 & -B^3 & B^2\\E^2 & B^3 &0 & -B^1\\E^3 & -B^2 & B^1 &0\\}\\
  {F_\mu}^\nu=g_{\mu\sigma}F^{\sigma\nu}                 & =g_{\mu0}F^{0\nu}+g_{\mu1}F^{1\nu}+g_{\mu2}F^{2\nu}+g_{\mu3}F^{3\nu}                                                                      \\
  {F_\mu}^\nu=\bmqty{g_{\mu\sigma}}\bmqty{F^{\sigma\nu}} & =\Pmqty{1                                                                                                            & 0    & 0    & 0    \\0&-1&0&0\\0&0&-1&0\\0&0&0&-1}\Pmqty{0 & -E^1 & -E^2 & -E^3 \\E^1 &0 & -B^3 & B^2\\E^2 & B^3 &0 & -B^1\\E^3 & -B^2 & B^1 &0\\}\\
                                                         & =\Pmqty{0                                                                                                            & -E^1 & -E^2 & -E^3 \\-E^1 &0&B^3&-B^2\\-E^2 & -B^3 &0 & B^1\\-E^3 & B^2 & -B^1 &0\\} \\
  F_{\mu\nu}=g_{\sigma\nu}{F_\mu}^\sigma                 & =g_{\nu0}{F_\mu}^0+g_{\nu1}{F_\mu}^1+g_{\nu2}{F_\mu}^2+g_{\nu3}{F_\mu}^3=\bmqty{{F_\mu}^\sigma}\bmqty{g_{\sigma\nu}}                      \\
  F_{\mu\nu}=\bmqty{{F_\mu}^\sigma}\bmqty{g_{\sigma\nu}} & =\Pmqty{0                                                                                                            & -E^1 & -E^2 & -E^3 \\-E^1 &0&B^3&-B^2\\-E^2 & -B^3 &0 & B^1\\-E^3 & B^2 & -B^1 &0\\}\Pmqty{1 & 0    & 0    & 0    \\0&-1&0&0\\0&0&-1&0\\0&0&0&-1} \\
                                                         & =\Pmqty{0                                                                                                            & E^1  & E^2  & E^3  \\-E^1 &0&-B^3&B^2\\-E^2 & B^3 &0 &-B^1\\-E^3 &-B^2 & B^1 &0\\} \\
\end{align*}




\begin{align*}
  \qq*{Lemma: for a given $\mu$,}\partial_\sigma F^{\mu\sigma} & = \partial_\sigma \pqty{\partial^\mu A^\sigma-\partial^\sigma A^\mu}        \\
                                                               & = \partial_\sigma\partial^\mu A^\sigma-\partial_\sigma\partial^\sigma A^\mu \\
                                                               & = \partial_\mu\partial^\mu A^\mu-\partial_\mu\partial^\mu A^\mu=0
\end{align*}

\begin{align*}
  \tilde{T}^{00} & =F^{0\sigma}{F_\sigma}^0+\frac{1}{4}F^{\alpha\beta}F_{\alpha\beta}\qq{where}g^{00}=1                          \\
                 & =F^{00}{F_0}^0+F^{01}{F_1}^0+F^{02}{F_2}^0+F^{03}{F_3}^0+\frac{1}{4}F^{\alpha\beta}F_{\alpha\beta}            \\
                 & =0^2+(E^1)^2+(E^2)^2+(E^3)^2+\frac{1}{4}F^{\alpha\beta}F_{\alpha\beta}                                        \\
                 & =\vb{E}\vdot\vb{E}+\frac{1}{4}F^{\alpha\beta}F_{\alpha\beta}                                                  \\
                 & =\vb{E}\vdot\vb{E}+\frac{1}{4}\pqty{F^{00}F_{00}+F^{01}F_{01}+F^{02}F_{02}+F^{03}F_{03}}                      \\
                 & +\frac{1}{4}\pqty{F^{10}F_{10}+F^{11}F_{11}+F^{12}F_{12}+F^{13}F_{13}}                                        \\
                 & +\frac{1}{4}\pqty{F^{20}F_{20}+F^{21}F_{21}+F^{22}F_{22}+F^{23}F_{23}}                                        \\
                 & +\frac{1}{4}\pqty{F^{30}F_{30}+F^{31}F_{31}+F^{32}F_{32}+F^{33}F_{33}}                                        \\
                 & =\vb{E}\vdot\vb{E}+\frac{1}{4}\pqty{-\vb{E}\vdot\vb{E}-\vb{E}\vdot\vb{E}+\vb{B}\vdot\vb{B}+\vb{B}\vdot\vb{B}} \\
                 & =\frac{1}{2}\pqty{\vb{E}\vdot\vb{E}+\vb{B}\vdot\vb{B}}=\frac{1}{2}\pqty{\vb{E}^2+\vb{B}^2}\qquad\blacksquare  \\
\end{align*}

\begin{align*}
  \tilde{T}^{i0} & =F^{i\sigma}{F_\sigma}^0\qq{where}g^{i0}=0                                  \\
  \tilde{T}^{i0} & =F^{i1}{F_1}^0+F^{i2}{F_2}^0+F^{i3}{F_3}^0\qq{where}{F_0}^0=0               \\
  \tilde{T}^{10} & =F^{12}{F_2}^0+F^{13}{F_3}^0\qq{where}F^{11}=0                              \\
                 & =B^3E^2-B^2E^3=-\pqty{\vb{B}\cross\vb{E}}^1\qq{recall}xyzzy                 \\
  \tilde{T}^{20} & =F^{21}{F_1}^0+F^{23}{F_3}^0\qq{where}F^{22}=0                              \\                  
                 & =-B^3E^1+B^1E^3=-\pqty{\vb{B}\cross\vb{E}}^2                                \\                  
  \tilde{T}^{30} & =F^{31}{F_1}^0+F^{32}{F_2}^0\qq{where}F^{33}=0                              \\                
                 & =B^2E^1-B^1E^2=-\pqty{\vb{B}\cross\vb{E}}^3                                 \\                
  \tilde{T}^{i0} & =-\pqty{\vb{B}\cross\vb{E}}^i=\pqty{\vb{E}\cross\vb{B}}^i\qquad\blacksquare \\
\end{align*}




\end{document}