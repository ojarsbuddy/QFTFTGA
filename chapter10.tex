\documentclass{article}
\usepackage{amsmath,amssymb,physics,xfrac,nicefrac}
\usepackage{standalone}
\usepackage[left=1.5cm,top=1.5cm,right=1.5cm,bottom=1.5cm]{geometry}
\usepackage[indentfirst=true]{quoting}
\usepackage{xcolor, soul}
\definecolor{HighlightGray}{cmyk}{0,0,0,0.07}
\sethlcolor{HighlightGray}
 
\title{Quantum Field Theory Problem Sets}
\author{John Bortins}
 
\begin{document}

\maketitle{}

\section*{Problem 10.1}
\begin{quoting}
    \hl{Show that $[\phi(x),P^\alpha]=i\partial^\alpha \phi(x)$, where $P^\alpha$ is the conserved charges from spacetime translation (10.28).}
\end{quoting}


If we can use equation 10.25, which is \[[\hat{Q}_{N},\hat{\phi}]=-iD\hat{\phi},\] then \[[\hat{\phi},\hat{Q}_{N}]=iD\hat{\phi}.\]

Eq.10.28 defines the conserved charges that arise from the conserved current $J_N^\mu=\alpha_\nu T^{\mu\nu}$: \[P^\alpha=\int \dd[3]{x}T^{0\alpha} .\]

So for the field $\phi(x)$ use the conserved charge $P^\alpha$ to generate the symmetry transformation
\begin{align*}
    [\phi(x),P^\alpha] & =iD\phi(x)                                         \\
                       & =i\omega_{\alpha\nu} x^\nu \partial^\alpha \phi(x)
\end{align*}



\section*{Problem 10.2}
\begin{quoting}
    \hl{Consider a system characterized by $N$ fields $\phi_1,\dots,\phi_N$. The Lagrangian density is then\\ $\mathcal{L}(\phi_1,\dots,\phi_N;\partial_\mu \phi_1,\dots,\partial_\mu \phi_N;x^\mu)$. Show that the Noether current is $J^\mu=\sum_\alpha \Pi_\alpha^\mu  D\phi^\alpha-W^\mu(x)$, (10.41),\\ where $D\mathcal{L}=\partial_\mu W^\mu$.}
\end{quoting}


\section*{Problem 10.3}
\begin{quoting}
    \hl{For the Lagrangian $\mathcal{L}=\frac{1}{2}(\partial_\mu \phi)^2-\frac{1}{2}m\phi^2$, (10.42), evaluate $T^{\mu\nu}$ and show that $T^{00}$ agrees with what you would expect from the Hamiltonian for this Lagrangian. Show that $\partial_\mu T^{\mu\nu}=0$. Derive expressions for $P^0=\int\dd[3]{x}T^{00}$ and $P^i=\int\dd[3]{x}T^{0i}$.}
\end{quoting}


\section*{Problem 10.4}
\begin{quoting}
    \hl{For the Lagrangian $\mathcal{L}=-\frac{1}{4}F_{\mu\nu}F^{\mu\nu}=\frac{1}{2}(\vb{E}^2-\vb{B}^2)$, (10.43), show that $\Pi^{\sigma\rho}\equiv\pdv{\mathcal{L}}{(\partial_\sigma A_\rho)}=-F^{\sigma\rho}$, (10.44). Hence show that the energy-momentum tensor $T^\mu_\nu=\Pi^{\mu\sigma} \partial_\nu A_\sigma-\delta^\mu_\nu \mathcal{L}$, (10.45), can be written as\\ $T^{\mu\nu}=-F^{\mu\sigma}\partial^\nu A_\sigma+\frac{1}{4}g^{\mu\nu}F^{\alpha\beta}F_{\alpha\beta}$, (10.46). This tensor is not symmetric but we can symmetrize it by adding $\partial_\lambda X^{\lambda\mu\nu}$ where $X^{\lambda\mu\nu}=F^{\mu\lambda}A^\nu$. Show that $X^{\lambda\mu\nu}=-X^{\mu\lambda\nu}$. Show further that the symmetrized energy-momentum tensor $\tilde{T}^{\mu\nu}=F^{\mu\sigma}{F_\sigma}^\nu+\frac{1}{4}g^{\mu\nu}F^{\alpha\beta}F_{\alpha\beta}$, (10.47). Hence show that $\tilde{T}^{00}=\frac{1}{2}(\vb{E}^2+\vb{B}^2)$, the energy density in the electromagnetic field, and $\tilde{T}^{i0}=(\vb{E}\cross\vb{B})^i$, which is the Poynting vector and describes the energy flow.}
\end{quoting}

\end{document}