\documentclass{article}
\usepackage{amsmath,amssymb,physics,xfrac,nicefrac}
\usepackage{standalone}
\usepackage[left=1.5cm,top=1.5cm,right=1.5cm,bottom=1.5cm]{geometry}
\usepackage[indentfirst=true]{quoting}
\usepackage{xcolor, soul}
\usepackage{enumitem}
\definecolor{HighlightGray}{cmyk}{0,0,0,0.07}
\sethlcolor{HighlightGray}
 
\title{Quantum Field Theory Problem Sets}
\author{John Bortins}
 
\begin{document}

\maketitle{}

\section*{\hl{Canonical quantization for a scalar field:}}
\begin{enumerate}[label=\Roman*.]
    \item \hl{Write down a Lagrangian density $\mathcal{L}$.}
    \item \hl{Evaluate the momentum density $\Pi^\mu (x)=\pdv*{\mathcal{L}}{(\partial_\mu \phi)}$ and the Hamiltonian density}
          \[\mathcal{H}=\Pi^0 \partial_0 \phi-\mathcal{L}.\]
    \item \hl{Turn fields into operators and enforce the commutation relation $\bqty{\hat{\phi}(x),\hat{\Pi}^0 (y)}=i\delta^{(3)}(\vb{x}-\vb{y})$ at equal times.}
    \item \hl{Express the field in terms of a mode expansion of the form}
          \[\hat{\phi}(x)=\int\frac{\dd[3]{\vb{p}}}{(2\pi)^{\frac{3}{2}}}\frac{1}{(2E_{\vb{p}})^{\frac{1}{2}}}\pqty{\hat{a}_{\vb{p}}e^{-i\vb{p}\vdot\vb{x}}+\hat{a}^\dagger_{\vb{p}}e^{i\vb{p}\vdot\vb{x}}} \]
          \hl{where $E_{\vb{p}}=+(\vb{p}^2+m^2)^{\frac{1}{2}}$.}
    \item \hl{Evaluate $\hat{\mathcal{H}}$ and normal order the result, leading to an expression like:}
          \[ N\bqty{\hat{\mathcal{H}}}=\int\dd[3]{\vb{p}}E_{\vb{p}}\hat{a}^\dagger_{\vb{p}}\hat{a}_{\vb{p}}.\]
\end{enumerate}



\section*{Problem 11.1}
\begin{quoting}
    \hl{One of the criteria we had for a theory of a successful scalar field ws that the commutator for space-like separations would be zero. Let's see if our scalar field has this feature. Show that}\[\bqty{\hat{\phi}(x),\hat{\phi}(y)}=\int\frac{\dd[3]{p}}{(2\pi)^3}\frac{1}{2E_{\vb{p}}}\pqty{e^{-ip\vdot(x-y)}+e^{-ip\vdot(y-x)}}\qquad(11.51).\]\hl{For space-like separation we are able to swap $(y-x)$ in the second term to $(x-y)$. This gives us zero, as required.}
\end{quoting}

\section*{Problem 11.2}
\begin{quoting}
    \hl{Show that, at equal times $x^0=y^0$,}\[\bqty{\hat{\phi}(x),\hat{\Pi}^0(y)}=\frac{i}{2}\int\frac{\dd[3]{p}}{(2\pi)^3}\pqty{e^{i\vb{p\vdot(\vb{x}-\vb{y})}}+e^{-i\vb{p\vdot(\vb{x}-\vb{y})}}}\qquad(11.52).\]\hl{In this expression there's nothing stopping us swapping the sign of $\vb{p}$ in the second term, and show that this leads to}\[\bqty{\hat{\phi}(x),\hat{\Pi}^0(y)}=i\delta^{(3)}(\vb{x}-\vb{y})\qquad(11.53).\]
\end{quoting}


\end{document}