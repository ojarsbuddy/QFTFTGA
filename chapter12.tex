\documentclass{article}
\usepackage{amsmath,amssymb,physics,xfrac,nicefrac}
\usepackage{standalone}
\usepackage[left=1.5cm,top=1.5cm,right=1.5cm,bottom=1.5cm]{geometry}
\usepackage[indentfirst=true]{quoting}
\usepackage{xcolor, soul}
\usepackage{enumitem}
\definecolor{HighlightGray}{cmyk}{0,0,0,0.07}
\sethlcolor{HighlightGray}
 
\title{Quantum Field Theory Problem Sets}
\author{John Bortins}
 
\begin{document}

\maketitle{}



\section*{Problem 12.1}
\begin{quoting}
    \hl{Fill in the missing steps of the algebra that led to eqn.12.6.}
\end{quoting}

\begin{align*}
    H           & =\int \dd[3]{x}\mathcal{H} \qq{(5.44)}                                                                                       \\
    \mathcal{H} & =\sum_\sigma \Pi^0_\sigma (x)\partial_0 \sigma(x)-\mathcal{L}                                                                \\
                & = \partial_0 \psi^\dagger (x)\partial_0 \psi(x)+\grad\psi^\dagger (x)\vdot\grad\psi(x)+m^2 \psi^\dagger(x)\psi(x)\qq{(12.3)} \\
\end{align*}

\begin{align*}
    \hat{\psi}(x)                       & =\int \frac{\dd[3]{p}}{(2\pi)^\frac{3}{2}}  \frac{1}{(2E_{\vb{p}})^{\frac{1}{2}}}  \pqty{\hat{a}_{\vb{p}}e^{-ip\vdot x}+\hat{b}^\dagger_{\vb{p}}e^{ip\vdot x}} \qq{(12.5)} \\
    \partial_\mu \hat{\psi}(x)          & =\int \frac{\dd[3]{p}}{(2\pi)^\frac{3}{2}}  \frac{1}{(2E_{\vb{p}})^{\frac{1}{2}}} (-ip) \pqty{\hat{a}_{\vb{p}}e^{-ip\vdot x}-\hat{b}^\dagger_{\vb{p}}e^{ip\vdot x}}        \\
    \hat{\psi}^\dagger (x)              & =\int \frac{\dd[3]{p}}{(2\pi)^\frac{3}{2}}  \frac{1}{(2E_{\vb{p}})^{\frac{1}{2}}}  \pqty{\hat{a}^\dagger_{\vb{p}}e^{-ip\vdot x}+\hat{b}_{\vb{p}}e^{ip\vdot x}} \qq{(12.5)} \\
    \partial_\mu \hat{\psi}^\dagger (x) & =\int \frac{\dd[3]{p}}{(2\pi)^\frac{3}{2}}  \frac{1}{(2E_{\vb{p}})^{\frac{1}{2}}} (-ip) \pqty{\hat{a}^\dagger_{\vb{p}}e^{-ip\vdot x}-\hat{b}_{\vb{p}}e^{ip\vdot x}}        \\
\end{align*}

Let's have a look at the last term.

\begin{align*}
    \psi^\dagger(x)\psi(x) & =\int \frac{\dd[3]{p}}{(2\pi)^\frac{3}{2}}  \frac{1}{(2E_{\vb{p}})^{\frac{1}{2}}}  \pqty{\hat{a}_{\vb{p}}e^{-ip\vdot x}+\hat{b}^\dagger_{\vb{p}}e^{ip\vdot x}}    \int \frac{\dd[3]{q}}{(2\pi)^\frac{3}{2}}  \frac{1}{(2E_{\vb{q}})^{\frac{1}{2}}}  \pqty{\hat{a}^\dagger_{\vb{q}}e^{-iq\vdot x}+\hat{b}_{\vb{q}}e^{iq\vdot x}}                                                                                             \\
                           & =\int \frac{\dd[3]{p}}{(2\pi)^\frac{3}{2}}  \frac{1}{(2E_{\vb{p}})^{\frac{1}{2}}} \int \frac{\dd[3]{q}}{(2\pi)^\frac{3}{2}}  \frac{1}{(2E_{\vb{q}})^{\frac{1}{2}}} \pqty{\hat{a}_{\vb{p}}e^{-ip\vdot x}+\hat{b}^\dagger_{\vb{p}}e^{ip\vdot x}}      \pqty{\hat{a}^\dagger_{\vb{q}}e^{-iq\vdot x}+\hat{b}_{\vb{q}}e^{iq\vdot x}}                                                                                             \\
                           & =\frac{1}{(2\pi)^3}\int   \frac{\dd[3]{p}}{(2E_{\vb{p}})^{\frac{1}{2}}} \int   \frac{\dd[3]{q}}{(2E_{\vb{q}})^{\frac{1}{2}}} \pqty{\hat{a}_{\vb{p}}\hat{a}^\dagger_{\vb{q}}e^{-ip\vdot x}e^{-iq\vdot x}+\hat{a}_{\vb{p}}\hat{b}^\dagger_{\vb{q}}e^{-ip\vdot x}e^{iq\vdot x}+\hat{b}^\dagger_{\vb{p}}\hat{a}^\dagger_{\vb{q}}e^{ip\vdot x}e^{-iq\vdot x}+\hat{b}^\dagger_{\vb{p}}\hat{b}_{\vb{q}}e^{ip\vdot x}e^{iq\vdot x}} \\
\end{align*}

\section*{Problem 12.2}
\begin{quoting}
    \hl{Evaluate (a) $\bqty{\hat{\psi}(x),\hat{\psi}^\dagger (y)}$ and (b) $\bqty{\hat{\Psi}(x),\hat{\Psi}^\dagger (y)}$ using the appropriate mode expansions in the text.}
\end{quoting}


\section*{Problem 12.3}
\begin{quoting}
    \hl{Consider the Lagrangian for two scalar fields from Section 7.5.}
    \begin{enumerate}[label=(\alph*)]
        \item \hl{Evaluate $\bqty{\hat{Q}_N,\hat{\phi}_1}$}
        \item \hl{and $\bqty{\hat{Q}_N,\hat{\phi}_2}$, where $\hat{Q}_N$ is the Noether charge. \emph{Hint: You could do this by brute force and evaluate $\hat{Q}_N$ and then find the commutator. A preferable method is just to use : $\bqty{\hat{Q}_N,\hat{\phi}_i}=-iD\hat{\phi}_i$ from Chapter 10.}}
        \item \hl{Use these results to show that $\bqty{\hat{Q}_N,\hat{\psi}}=\hat{\psi}$.}
    \end{enumerate}
\end{quoting}


\section*{Problem 12.4}
\begin{quoting}
    \hl{Consider the theory described by the Lagrangian in eqn 12.32.

        Use Noether's theorem in the form $\bqty{\hat{Q}_N,\hat{\phi}}=-iD\hat{\phi}$ to provide an alternative derivation of eqn 12.35.}
\end{quoting}


\section*{Problem 12.5}
\begin{quoting}
    \hl{Apply the Euler-Lagrange equations to eqn 12.24 and show that it yields $E_{\vb{p}}=\vb{p}^2 /2m$ when $V=0$.}
\end{quoting}


\section*{Problem 12.6}
\begin{quoting}
    \hl{Find the Noether current for the Lagrangian in eqn 12.24. Check that it is what you expect for non-relativistic quantum mechanics}
\end{quoting}


\section*{Problem 12.7}
\begin{quoting}
    \hl{Consider the complex scalar field again. The internal transformation operator may be written

    $\hat{U}(\alpha)=e^{i\hat{Q}_{Nc}\alpha}$, where $\hat{Q}_{Nc}$ is the conserved number charge operator. Show that $\hat{U}^\dagger (\alpha)\hat{\psi}(x)\hat{U}(\alpha)=e^{i\alpha}\hat{\psi}(x)$.}
\end{quoting}



\end{document}