\documentclass{article}
\usepackage{amsmath,amssymb,physics,xfrac,nicefrac}
\usepackage{standalone}
\usepackage[left=1.5cm,top=1.5cm,right=1.5cm,bottom=1.5cm]{geometry}
\usepackage[indentfirst=true]{quoting}
\usepackage{xcolor, soul}
\usepackage{enumitem}
\definecolor{HighlightGray}{cmyk}{0,0,0,0.07}
\sethlcolor{HighlightGray}
 
\title{Quantum Field Theory Problem Sets}
\author{John Bortins}
 
\begin{document}

\maketitle{}


\section*{Problem 13.1}
\begin{quoting}
  \hl{(a) Show that the conserved charge in eqn. 13.16 may be written}
  \[\vb{\hat{Q}}_{Nc}=\int\dd[3]{p}\vb{\hat{A}}^\dagger_{\vb{p}}\vb{J}\vb{\hat{A}}_{\vb{p}} \]
  \hl{where  $\vb{\hat{A}}_{\vb{p}}=(\hat{a}_{1\vb{p}},\hat{a}_{2\vb{p}},\hat{a}_{3\vb{p}})$ and  $\vb{J}=(J_x,J_y,J_z)$ are the spin-1 angular momentum matrices from chapter 9.}

  \hl{(b) Use the transformations from Exercise 3.3 to find the form of the angular momentum matrices appropriate to express the charge as  $\vb{\hat{Q}}_{Nc}=\int\dd[3]{p}\vb{\hat{B}}^\dagger_{\vb{p}}\vb{J}\vb{\hat{B}}_{\vb{p}}$ where $\vb{\hat{B}}_{\vb{p}}=(\hat{b}_{1\vb{p}},\hat{b}_{0\vb{p}},\hat{b}_{-1\vb{p}})$.}
\end{quoting}



\begin{align*}
  \hat{Q}^a_{Nc}                                                                        & =-i\int\dd[3]{p}\epsilon^{abc}\hat{a}^\dagger_{b\vb{p}}\hat{a}_{c\vb{p}}\qquad\qq{(13.16)}                         \\
  \qq*{Consider each component}\epsilon^{1bc}\hat{a}^\dagger_{b\vb{p}}\hat{a}_{c\vb{p}} & =\epsilon^{123}\hat{a}^\dagger_{2\vb{p}}\hat{a}_{3\vb{p}}+\epsilon^{132}\hat{a}^\dagger_{3\vb{p}}\hat{a}_{2\vb{p}} \\
                                                                                        & =\hat{a}^\dagger_{2\vb{p}}\hat{a}_{3\vb{p}}-\hat{a}^\dagger_{3\vb{p}}\hat{a}_{2\vb{p}}                             \\
  \epsilon^{2bc}\hat{a}^\dagger_{b\vb{p}}\hat{a}_{c\vb{p}}                              & =\hat{a}^\dagger_{3\vb{p}}\hat{a}_{1\vb{p}}-\hat{a}^\dagger_{1\vb{p}}\hat{a}_{3\vb{p}}                             \\
                                                                                        & =-\hat{a}^\dagger_{1\vb{p}}\hat{a}_{3\vb{p}}+\hat{a}^\dagger_{3\vb{p}}\hat{a}_{1\vb{p}}                            \\
  \epsilon^{3bc}\hat{a}^\dagger_{b\vb{p}}\hat{a}_{c\vb{p}}                              & =\hat{a}^\dagger_{1\vb{p}}\hat{a}_{2\vb{p}}-\hat{a}^\dagger_{2\vb{p}}\hat{a}_{1\vb{p}}                             \\
\end{align*}

\[\vb{\hat{A}}^\dagger_{\vb{p}}J_a\vb{\hat{A}}_{\vb{p}}=\vb{\hat{A}}^\dagger_{\vb{p}}\begin{pmatrix}
    J_{11} & J_{12} & J_{13} \\
    J_{21} & J_{22} & J_{23} \\
    J_{31} & J_{32} & J_{33} \\
  \end{pmatrix}\begin{pmatrix}
    \hat{a}_{1\vb{p}} \\
    \hat{a}_{2\vb{p}} \\
    \hat{a}_{3\vb{p}} \\
  \end{pmatrix}=\vb{\hat{A}}^\dagger_{\vb{p}}\begin{pmatrix}
    J_{11}\hat{a}_{1\vb{p}} + J_{12}\hat{a}_{2\vb{p}} + J_{13}\hat{a}_{3\vb{p}} \\
    J_{21}\hat{a}_{1\vb{p}} + J_{22}\hat{a}_{2\vb{p}} + J_{23}\hat{a}_{3\vb{p}} \\
    J_{31}\hat{a}_{1\vb{p}} + J_{32}\hat{a}_{2\vb{p}} + J_{33}\hat{a}_{3\vb{p}} \\
  \end{pmatrix}\qq{where} J_{ii}=0\]

\[\vb{\hat{A}}^\dagger_{\vb{p}}J_a\vb{\hat{A}}_{\vb{p}}=\vb{\hat{A}}^\dagger_{\vb{p}}\begin{pmatrix}
    0      & J_{12} & J_{13} \\
    J_{21} & 0      & J_{23} \\
    J_{31} & J_{32} & 0      \\
  \end{pmatrix}\begin{pmatrix}
    \hat{a}_{1\vb{p}} \\
    \hat{a}_{2\vb{p}} \\
    \hat{a}_{3\vb{p}} \\
  \end{pmatrix}=\begin{pmatrix}
    \hat{a}^\dagger_{1\vb{p}} &
    \hat{a}^\dagger_{2\vb{p}} &
    \hat{a}^\dagger_{3\vb{p}}
  \end{pmatrix}\begin{pmatrix}
    J_{12}\hat{a}_{2\vb{p}} + J_{13}\hat{a}_{3\vb{p}} \\
    J_{21}\hat{a}_{1\vb{p}} + J_{23}\hat{a}_{3\vb{p}} \\
    J_{31}\hat{a}_{1\vb{p}} + J_{32}\hat{a}_{2\vb{p}} \\
  \end{pmatrix}\]

\begin{align*}
  \vb{\hat{A}}^\dagger_{\vb{p}}J_a\vb{\hat{A}}_{\vb{p}} & =
  \hat{a}^\dagger_{1\vb{p}}J_{12}\hat{a}_{2\vb{p}} + \hat{a}^\dagger_{1\vb{p}}J_{13}\hat{a}_{3\vb{p}}
  +\hat{a}^\dagger_{2\vb{p}}J_{21}\hat{a}_{1\vb{p}} + \hat{a}^\dagger_{2\vb{p}}J_{23}\hat{a}_{3\vb{p}}
  +\hat{a}^\dagger_{3\vb{p}}J_{31}\hat{a}_{1\vb{p}} + \hat{a}^\dagger_{3\vb{p}}J_{32}\hat{a}_{2\vb{p}} \\
\end{align*}



\begin{align*}
  -\epsilon^{1bc}\hat{a}^\dagger_{b\vb{p}}\hat{a}_{c\vb{p}}  =-\hat{a}^\dagger_{2\vb{p}}\hat{a}_{3\vb{p}}+\hat{a}^\dagger_{3\vb{p}}\hat{a}_{2\vb{p}} & \implies J_x=i\begin{pmatrix}
    0 & 0 & 0  \\
    0 & 0 & -1 \\
    0 & 1 & 0  \\
  \end{pmatrix} \\
  -\epsilon^{2bc}\hat{a}^\dagger_{b\vb{p}}\hat{a}_{c\vb{p}}  =\hat{a}^\dagger_{1\vb{p}}\hat{a}_{3\vb{p}}-\hat{a}^\dagger_{3\vb{p}}\hat{a}_{1\vb{p}}  & \implies J_y=i\begin{pmatrix}
    0  & 0 & 1 \\
    0  & 0 & 0 \\
    -1 & 0 & 0 \\
  \end{pmatrix} \\
  -\epsilon^{3bc}\hat{a}^\dagger_{b\vb{p}}\hat{a}_{c\vb{p}}  =-\hat{a}^\dagger_{1\vb{p}}\hat{a}_{2\vb{p}}+\hat{a}^\dagger_{2\vb{p}}\hat{a}_{1\vb{p}} & \implies J_z=i\begin{pmatrix}
    0 & -1 & 0 \\
    1 & 0  & 0 \\
    0 & 0  & 0 \\
  \end{pmatrix} \\
\end{align*}

Which are the spin-1 matrices found in Chapter 9 but without the time components. $\qquad\blacksquare$





\section*{Problem 13.2}
\begin{quoting}
  \hl{(a)  Confirm that eqn 13.29 is the appropriate matrix to boost a particle along the $z$-direction.}

  \hl{(b)  Show that the boosted vectors in eqn 13.30 are still correctly normalized.}

  \hl{(c)  Consider the circular polarization vectors $\epsilon^{\mu*}_{\lambda=R}=-\frac{1}{\sqrt{2}}(0,1,i,0)$, $\epsilon^{\mu*}_{\lambda=L}=\frac{1}{\sqrt{2}}(0,1,-i,0)$, $\epsilon^{\mu*}_{\lambda=3}=-\frac{1}{\sqrt{2}}(0,0,0,1)$. Show that these are correctly normalized according to $g_{\mu\nu}\epsilon^{\mu*}_\lambda \epsilon^\nu_{\lambda'}=-\delta_{\lambda\lambda'}$.}
\end{quoting}

\section*{Problem 13.3}
\begin{quoting}
  \hl{Show that $P_L$ and $P_T$ are indeed projection operators by showing that $P^2=P$.}
\end{quoting}
\begin{align*}
  \vb{X}           & =\vb{X}_L+\vb{X}_T             \\
  X^i              & =X^i_L+X^i_T                   \\
  X^i              & =P^{ij}_LX^j+P^{ij}_TX^j       \\
  P^{ij}X^j        & =P^{ij}_LX^j+P^{ij}_TX^j       \\
  P^{ij}X^j        & =(P^{ij}_L+P^{ij}_T)X^j        \\
  P^{ij}           & =P^{ij}_L+P^{ij}_T=\delta_{ij} \\
  \qq*{Clearly}P^2 & =P\qquad\blacksquare
\end{align*}



\section*{Problem 13.4}
\begin{quoting}
  \hl{The Lagrangian for electromagnetism in vacuo is $\mathcal{L}=-\frac{1}{4}F^{\mu\nu}F_{\mu\nu}$. Show that this can be rewritten as}\[\mathcal{L}=\frac{1}{2}A^\mu P^T_{\mu\nu}\partial^2A^\nu.\]
  \hl{This shows that $\mathcal{L}$ only includes the transverse components of the field, squaring with the idea of electromagnetic waves only representing vibrations transverse to the direction of propagation.}
\end{quoting}


\end{document}