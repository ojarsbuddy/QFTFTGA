\documentclass[letterpaper]{article}

\usepackage{amsmath,amsthm,amssymb,physics,xfrac,nicefrac}
\usepackage{standalone}
\usepackage[left=1.5cm,top=1.5cm,right=1.5cm,bottom=1.5cm]{geometry}
\usepackage{xcolor, soul}
\usepackage{enumitem}

\theoremstyle{definition}
\newtheorem{prob}{Problem}[section]
\renewcommand{\qedsymbol}{\(\blacksquare\)}
 
\setcounter{section}{14}
\title{Quantum Field Theory Problem Set}
\author{John Bortins}
 
\begin{document}

\maketitle{}


\begin{prob}
  Fill in the algebra leading to eqn 14.32.
\end{prob}
The canonical quantization machinery

Step I: Write down a classical Lagrangian density in terms of fields.\[\mathcal{L}=-\frac{1}{4}F_{\mu\nu}F^{\mu\nu}=-\frac{1}{4}\pqty{\partial_\mu A_\nu-\partial_\nu A_\mu}\pqty{\partial^\mu A^\nu-\partial^\nu A^\mu}\]

Step II: Calculate the momentum density and Hamiltonian density in terms of fields.

We chose the Lorenz gauge \(\partial_\mu A^\mu=0.\) With \(A^0=0\) we see \(\partial_i A^i=\div \vb*{A}=0.\)
\begin{align*}
  \Pi^{\mu\nu} & =\pdv{\mathcal{L}}{(\partial_\mu A_\nu)}=-F^{\mu\nu}=-\pqty{\partial^\mu A^\nu-\partial^\nu A^\mu} \\
  \Pi^{00}     & =-\pqty{\partial^0 A^0-\partial^0 A^0}=0                                                           \\
  \Pi^{0j}     & =-\pqty{\partial^0 A^j-\partial^j A^0}=-\partial^0 A^j=-\pdv{A^j}{x_0}=-\pdv{A^j}{t}=-E^j          \\
  \Pi^{i0}     & =-\pqty{\partial^i A^0-\partial^0 A^i}= \partial^0 A^i=\pdv{A^i}{x_0}=\pdv{A^i}{t}=E^i             \\
\end{align*}
\begin{align*}
  \mathcal{H} & ={\pdv{\mathcal{L}}{(\partial_0 A_\nu)}}   (\partial_0 A_\nu)-\mathcal{L} \\
  \mathcal{H} & =\frac{1}{2}\pqty{\vb{E}^2+\vb{B}^2}                                      \\
\end{align*}

Step III: Treat the fields and momentum density as operators. Impose commutation relations to make all quantum mechanical.

Put hats on the fields \(\hat{A}^\mu\) and momentum density \(\hat{\Pi}^{\mu\nu}\).\[\comm{\hat{A}^i(\vb{x})}{\hat{E}^j(\vb{y})}=\text{i}\int\frac{\dd[3]{p}}{(2\pi)^3}\text{e}^{\text{i}\vb*{p}\cdot(\vb*{x}-\vb*{y})}\pqty{\delta^{ij}-\frac{p^i p^j}{p^2}}=\text{i}\delta^{(3)}_{\text{tr}}(\vb*{x}-\vb*{y}).\qquad(14.30)\]

Step IV: Expand the fields as creation and annihilation operators.\[\hat{A}^\mu(x)=\int\frac{\dd[3]{p}}{(2\pi)^{\frac{3}{2}}}\frac{1}{(2E_{\vb*{p}})^\frac{1}{2}}\sum^2_{\lambda=1}\pqty{\epsilon^\mu_\lambda(p)\hat{a}_{\vb*{p}\lambda}e^{-\text{i}p\vdot x}+\epsilon^{\mu*}_\lambda(p)\hat{a}^\dagger_{\vb*{p}\lambda}e^{\text{i}p\vdot x}}.\qquad(14.31)\]

\begin{proof}
  \begin{align*}
    \qq*{Step V: Apply the \emph{normal ordering} interpretation.} \\
  \end{align*}

\end{proof}



\begin{prob}
  \emph{A demonstration that the photon has spin-1, with only two spin polarizations.}

  A photon \(\gamma\) propagates with momentum \(q^\mu=(|\vb*{q}|,0,0,|\vb*{q}|)\). Working with a basis where the two transverse photon polarizations are \(\epsilon^\mu_{\lambda=1}(q)=(0,1,0,0)\) and \(\epsilon^\mu_{\lambda=2}(q)=(0,0,1,0)\), it may be shown, using Noether's theorem, that the operator \(\hat{S}^z\), whose eigenvalue is the \(z-\)component spin angular momentum of the photon, obeys the commutator relation \[\bqty{\hat{S}^z,\hat{a}^\dagger_{\vb*{q}\lambda}}=\text{i}\epsilon^{\mu=1*}_\lambda(q)\hat{a}^\dagger_{\vb*{q}\lambda=2}-\text{i}\epsilon^{\mu=2*}_\lambda(q)\hat{a}^\dagger_{\vb*{q}\lambda=1}.\qq{(14.36)}\]

  (i) Define creation operators for the circular polatizations via
  \begin{align*}\hat{b^\dagger_{\vb*{q}\text{R}}} & =-\frac{1}{\sqrt{2}}\pqty{\hat{a}^\dagger_{\vb*{q}1}+\text{i}\hat{a}^\dagger_{\vb*{q}2}},             \\
    \hat{b^\dagger_{\vb*{q}\text{L}}} & =\frac{1}{\sqrt{2}}\pqty{\hat{a}^\dagger_{\vb*{q}1}-\text{i}\hat{a}^\dagger_{\vb*{q}2}}.\qquad(14.37)
  \end{align*}


  Show that\begin{align*}\bqty{\hat{S}^z,\hat{b}^\dagger_{\vb*{q}\text{R}}}&=\hat{b}^\dagger_{\vb*{q}\text{R}},\\\bqty{\hat{S}^z,\hat{b}^\dagger_{\vb*{q}\text{L}}}&=-\hat{b}^\dagger_{\vb*{q}\text{L}}.\qquad(14.38)\end{align*}

  (ii) Consider the opertion of \(\hat{S}^z\) on a state \(\ket*{\gamma_{\vb*{q}\lambda}}=\hat{b}^\dagger_{\vb*{q}\lambda}\ket*{0}\) containing a single photon propagating along \(z\):\[\hat{S}^z\ket*{\gamma\lambda}=\hat{S}^z\hat{b}^\dagger_{\vb*{q}\lambda}\ket*{0}\qcomma\lambda=\text{R, L.}\qquad(14.39)\]

  Use the results of (i) to argue that the projection of the photon spin along its direction of propagation must be \(\hat{S}^z=\pm 1.\)

  \emph{See Bjorken and Drell Chapter 14 for the full version of this argument.}
\end{prob}

\begin{proof}Part (i)
  \begin{align*}
    \bqty{\hat{S}^z,\hat{b}^\dagger_{\vb*{q}R}} & =\bqty{\hat{S}^z,-\frac{1}{\sqrt{2}}\pqty{\hat{a}^\dagger_{\vb*{q}1}+\text{i}\hat{a}^\dagger_{\vb*{q}2}}}=-\frac{1}{\sqrt{2}}\bqty{\hat{S}^z,\hat{a}^\dagger_{\vb*{q}1}}-\frac{\text{i}}{\sqrt{2}}\bqty{\hat{S}^z,\hat{a}^\dagger_{\vb*{q}2}}                                                                                  \\
                                                & =-\frac{1}{\sqrt{2}}\bqty{\text{i}\epsilon^{\mu=1*}_1(q)\hat{a}^\dagger_{\vb*{q}\lambda=2}-\text{i}\epsilon^{\mu=2*}_1(q)\hat{a}^\dagger_{\vb*{q}\lambda=1}}-\frac{\text{i}}{\sqrt{2}}\bqty{\text{i}\epsilon^{\mu=1*}_2(q)\hat{a}^\dagger_{\vb*{q}\lambda=2}-\text{i}\epsilon^{\mu=2*}_2(q)\hat{a}^\dagger_{\vb*{q}\lambda=1}} \\
                                                & =-\frac{1}{\sqrt{2}}\bqty{\text{i}\hat{a}^\dagger_{\vb*{q}2}+\hat{a}^\dagger_{\vb*{q}1}}=-\frac{1}{\sqrt{2}}\bqty{\hat{a}^\dagger_{\vb*{q}1}+\text{i}\hat{a}^\dagger_{\vb*{q}2}}                                                                                                                                               \\
                                                & =\hat{b}^\dagger_{\vb*{q}\text{R}}
  \end{align*}
  \begin{align*}
    \bqty{\hat{S}^z,\hat{b}^\dagger_{\vb*{q}L}} & =\bqty{\hat{S}^z,\frac{1}{\sqrt{2}}\pqty{\hat{a}^\dagger_{\vb*{q}1}-\text{i}\hat{a}^\dagger_{\vb*{q}2}}}=\frac{1}{\sqrt{2}}\bqty{\hat{S}^z,\hat{a}^\dagger_{\vb*{q}1}}-\frac{\text{i}}{\sqrt{2}}\bqty{\hat{S}^z,\hat{a}^\dagger_{\vb*{q}2}}                                                                                   \\
                                                & =\frac{1}{\sqrt{2}}\bqty{\text{i}\epsilon^{\mu=1*}_1(q)\hat{a}^\dagger_{\vb*{q}\lambda=2}-\text{i}\epsilon^{\mu=2*}_1(q)\hat{a}^\dagger_{\vb*{q}\lambda=1}}-\frac{\text{i}}{\sqrt{2}}\bqty{\text{i}\epsilon^{\mu=1*}_2(q)\hat{a}^\dagger_{\vb*{q}\lambda=2}-\text{i}\epsilon^{\mu=2*}_2(q)\hat{a}^\dagger_{\vb*{q}\lambda=1}} \\
                                                & =\frac{1}{\sqrt{2}}\bqty{\text{i}\hat{a}^\dagger_{\vb*{q}2}-\hat{a}^\dagger_{\vb*{q}1}}=-\frac{1}{\sqrt{2}}\bqty{\hat{a}^\dagger_{\vb*{q}1}-\text{i}\hat{a}^\dagger_{\vb*{q}2}}                                                                                                                                               \\
                                                & =-\hat{b}^\dagger_{\vb*{q}\text{L}}
  \end{align*}
\end{proof}

\begin{proof}Part (ii)
  \begin{align*}
    \bqty{\hat{S}^z,\hat{b}^\dagger_{\vb*{q}R}}                                             & =\hat{b}^\dagger_{\vb*{q}\text{R}}              \\
    \hat{S}^z\hat{b}^\dagger_{\vb*{q}R}-\hat{b}^\dagger_{\vb*{q}R}\hat{S}^z                 & =\hat{b}^\dagger_{\vb*{q}\text{R}}              \\
    \hat{S}^z\hat{b}^\dagger_{\vb*{q}R}\ket*{0}-\hat{b}^\dagger_{\vb*{q}R}\hat{S}^z\ket*{0} & =\hat{b}^\dagger_{\vb*{q}\text{R}}\ket*{0}      \\
    \hat{S}^z\ket*{\gamma R}                                                                & =\ket*{\gamma R}\qq{because}\hat{S}^z\ket*{0}=0 \\
  \end{align*}

  \begin{align*}
    \bqty{\hat{S}^z,\hat{b}^\dagger_{\vb*{q}L}}                                             & =-\hat{b}^\dagger_{\vb*{q}\text{L}}              \\
    \hat{S}^z\hat{b}^\dagger_{\vb*{q}L}-\hat{b}^\dagger_{\vb*{q}L}\hat{S}^z                 & =-\hat{b}^\dagger_{\vb*{q}\text{L}}              \\
    \hat{S}^z\hat{b}^\dagger_{\vb*{q}L}\ket*{0}-\hat{b}^\dagger_{\vb*{q}L}\hat{S}^z\ket*{0} & =-\hat{b}^\dagger_{\vb*{q}\text{L}}\ket*{0}      \\
    \hat{S}^z\ket*{\gamma L}                                                                & =-\ket*{\gamma L}\qq{because}\hat{S}^z\ket*{0}=0 \\
  \end{align*}

  \begin{align*}
    \hat{S}^z\ket*{\gamma \lambda} & =\pm\ket*{\gamma \lambda} \\
    \hat{S}^z                      & =\pm 1                    \\
  \end{align*}
\end{proof}

Errata fixed
\[\mathcal{L}=-\frac{1}{4}\pqty{\partial_\mu A_\nu-\partial_\nu A_\mu}\pqty{\partial^\mu A^\nu-\partial^\nu A^\mu}-J^\mu_{\text{em}}A_\mu \qquad(14.12)\]
\[\text{i}\partial^j\delta^{(3)}(\vb*{x}-\vb*{y}) \qquad(14.29)\]
\end{document}